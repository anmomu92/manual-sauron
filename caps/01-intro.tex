\chapter{Introducción}\label{sec:cap1}
SAURON es un proyecto académico para investigar mejoras en redes de interconexión de altas prestaciones utilizando el entorno de simulación que proporciona OMNeT++. Se trata de una red de computación de altas prestaciones basada en el framework OMNeT++. SAURON es un acrónimo de \emph{Simulador de ArqUitecturas de Red en OmNet++}.

Esta herramienta ha sido desarrollada en el departamento de sistemas informáticos de la Universidad de Castilla-La Mancha, en Albacete. Su desarrollo arrancó en el verano de 2011 como el Trabajo Fin de Grado de Pedro Yébenes Segura (programador principal), que fue supervisado por Jesús Escudero Sahuquillo, Pedro Javier García García y Francisco José Quiles Flor.

\section{Características}
SAURON modela las siguientes características:

\begin{itemize}
    \item \textbf{Topologías.} Meshes, tori, fat trees, KNS, DragonFly, SlimFly.
    \item \textbf{Algoritmos de encaminamiento.} DOR, destro, Hybrid-DOR, MIN routing for dragonfly topology, etc.
    \item \textbf{Esquemas de colas.} VOQnet, VOQsw, DBBM, OBQA, BBQ, Flow2SL, H2LQ, IODET, etc.
    \item \textbf{Calidad de servicio (QoS).} InfiniBand-based VLTables.
    \item \textbf{Tráfico basado en tráfico real.} Por medio de la biblioteca TraceLib.
\end{itemize}
